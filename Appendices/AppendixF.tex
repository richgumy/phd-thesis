\chapter{Dielectric Elastomer Actuator and Generator Validation}

\section{DEA Energy Density Calculations}
\label{DEA_Edensity}
For the following energy density calculations the device material is assumed a homogeneous Hookean solid and fringing effects are ignored by using the parallel plate capacitor assumption. First the mechanical energy density, $U_{mech}$ is calculated followed by the electrical energy density, $U_{elec}$. The difference is found to generate a 'total' energy density, $U_{tot}$. Note this is an estimation, for a more comprehensive and physically realistic calculation please refer to state-of-art literature.
\begin{equation}
	U_{mech} = 0.5 \cdot \sigma \cdot \varepsilon
\end{equation}
Where $\sigma$ is average stress during a maximum strain event, $\varepsilon$.
\begin{equation}
	U_{elec} = 0.5 \cdot \varepsilon_0 \cdot \varepsilon_r \cdot E^2
\end{equation}
Where $\varepsilon_0$ and $\varepsilon_r$ are vacuum and relative permittivity respectively, and $E$ is the electric field at maximum excitation strain.
\begin{equation}
	U_{tot} = U_{elec} - U_{mech}
\end{equation}

\section{$\Delta V$ Calculcation}
\label{appendix-F}
An estimate of the ideal voltage, $\Delta V$, generated from a circular compression load of radius, $r_L$, on a circular DE domain of radius, $r_{DE}$, with two conductive compliant electrodes covering the area $A_{DE}$ of the DE.

Several assumptions are made to simplify the calculation to be solved analytically:
\begin{enumerate}
	\item The capacitance of the unloaded system is made up of the base DE capacitance.
	\item The capacitance of the loaded system is a summation of the base DE capacitance and loaded area capacitance.
	\item The parallel plate capacitor assumptions are made for both DE and loaded area capacitances.
	\item The relative permittivity of the DE remains constant with changing strain.
\end{enumerate}

The dependent parameters voltage, $\Delta V$, and energy, $\Delta U$, generated by the DEG are a result of the change in the DE capacitance, the charge, $Q$, on the DE electrodes, and the physical parameters of the DE. The independent physical parameters of the DE system include:
\begin{itemize}
	\item $A_{DE}$ - area of the dielectric elastomer capacitance
	\item $d_i$ - relaxed DE thickness
	\item $\epsilon_r$ - relative permittivity of free space
	\item $V_i$ - initial priming voltage input
	\item $Q$ - DE charge
\end{itemize}
The parameters independent parameters for driving energy generation include:
\begin{itemize}
	\item $\varepsilon$ - compressive strain of the load applied
	\item $A_L$ - area of the applied load
\end{itemize}

Using Equations \ref{eqn:elastic-energy} - \ref{eqn:elec-energy2} as a basis for the DEG method, we form a formula for the voltage generated $\Delta V$, based on the priming voltage, $V_i$ and the stage two voltage, $V_{ii}$ as,
\begin{equation}
	\Delta V = V_{ii} - V_i = \frac{Q}{C_{i}} - \frac{Q}{C_{ii}}.
	\label{eqn:F1}
\end{equation}
Which then can be rearranged to obtain a formula for $\Delta V$ as a function of $C$ and $V_i$,
\begin{equation}
	\Delta V = V_i \left(\frac{C_{ii}}{C_i} - 1\right)
	\label{eqn:F2}
\end{equation}

For a load with a strain, $\varepsilon$, we get a compressed DE thickness $d_{ii}$ such that,

\begin{equation}
	d_{ii} = d_i(\varepsilon + 1).
	\label{eqn:F3}
\end{equation}

Substituting the parallel capacitance formulae for $C_i$ and $C_{ii}$ we get,

\begin{equation}
	C_i = \frac{\epsilon (A_{DE})}{d_i}
	\label{eqn:F4}
\end{equation}
and
\begin{equation}
	C_{ii} = \frac{\epsilon (A_{DE} - A_L)}{d_i} + \frac{\epsilon A_L}{d_{ii}}
	\label{eqn:F5}
\end{equation}

where $\epsilon = \epsilon_0\epsilon_r$. By substituting Equations \ref{eqn:F3} - \ref{eqn:F5} into Equation \ref{eqn:F2} we can obtain an analytical solution for the system. 

For a more accurate analytical solution utilising surface integrals and Maxwell's equations could be used on a case by case basis for different loads. However this approach has not been used as each deriving each analytical solution is cumbersome and not extendable for a range of load shapes that may be experienced in real world scenarios.


