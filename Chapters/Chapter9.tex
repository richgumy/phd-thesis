\chapter{\chapixname}
\label{chapter9}
Here we are. The final chapter. Time to reflect and get philosophical!

\section{Core Research Findings}
A literature review comparing human skin and muscle tissue to current soft pressure mapping and soft electroactive actuator technologies was completed. From this review it was determined that CBSR material would be suitable for a range of soft pressure mapping and actuation devices, because it's readily available, non-toxic, conductive, piezoresistive, mouldable, and highly deformable. Two specific technologies that emulate skin in terms of pressure mapping and muscle in terms of actuation were chosen, EIT-based pressure mapping and dielectric elastomer actuators. 
both of which use this  CBSR material as the electroactive polymer (EAP) component of of the sensor/actuator.

The first novel work explored the characterisation and selection of carbon black (CB) silicone rubber (SR) composites through several imaging techniques, including microscopy, scanning electron microscopy, and Raman spectroscopy. This imaging showed the carbon black micro-structure and how it was dispersed through a the micro-void filled elastomer composite. Then a macro-characterisation of the material was completed by determining several repeatable dynamic and transient characteristics by fitting several mathematical formulae to the repeated phenomena. This characterisation shows that the material has repeatable characteristics and elucidated how the material reacts to different time series tensile strain inputs and set a foundation for development of sensors and actuators for the material.

Using different configurations of the same CBSR material in Chapter \ref{chapter3}, electrical impedance tomography (EIT) was used to map dynamic changes in resistance within a CBSR sheet of material undergoing a series of compressive loads. A set of performance metrics were developed as tools to determine the quality of a reconstruction given a known load input. The resistive relaxation was quantified in the 2D EIT reconstruction and compared to a 1D equivalent resistive relaxation, to show that 1D data and characterisation could be extrapolated to handle 2D problems. A model was then created which translates changes in conductivity to force elements. 

To allow EIT-based pressure mapping to be used in a variety of portable/wearable, small scale applications a open-source circuit was developed. In parallel alongside a Cartesian force applicator was developed especially to capture compressive load data so that such pressure sensor could be characterised for a range of different sensor domain materials.

Dielectric elastomer actuators are soft actuators that can emulate human muscle tissue function. To enhance the functionality of these actuator devices dielectric elastomer actuator technology was combined with the EIT-based sensor developed in this thesis to give a sense of touch to the actuator in the form of EIT-based pressure mapping. Such a DEA-EIT actuator-sensor device has not been disseminated, so a patent towards this technology has been filed provisionally. This work has investigated the complications with integrating the two technologies such as the inherent trade off between pressure mapping resolution and actuation performance with the current construction of the device.

During the validation and characterisation of the DEA-EIT device this work shows the formation of an unintentional dielectric elastomer generator (DEG) and the ability to map and localise compressive events triggering the generation of energy. The energy generated by such a device in practice is dependent on the topology of the device. Optimisation of such DEG, has not yet been explored in this work.



\section{Future Work}
To move the technology developed in this work into real-world applications and further enhance its performance more research and development is required for Chapters \ref{chapter3} to \ref{chapter7}.

Although several methods of macro-modelling have been developed to represent a piezoresistive relationship in a conductive particle elastomer composite. There has been no evidence in literature of in-situ particle scale imaging of a CPEC specimen undergoing a known strain event. Which could be done using imaging devices such as transmission electron microscopy (TEM) and small angle x-ray scattering (SAXS). This would show researchers how the particles move within a strained material. If using SAXS an electrical current could also be applied to the material to determine whether an applied electric field changes the motion of the particles signifcantly and explains other electromechanical phenomena measured in the material. In parallel with this particle modelling, more macro-level modelling needs to be pursued to mitigate time dependent phenomena slowing down the response time of the sensor.

One of the main issues determined with the material is quantifying the degree of material homogeneity in terms of carbon black particle and void dispersion and how to determine the piezoresistivity to calibrate based on non-homogeneity within the CBSR material volume. Depending on the location of a compressive load the detected change in resistance would vary massively. Determingin this dispersion could be done with a range of imaging techniques including TEM and SAXS.

Development of Electrical Impedance Tomography (EIT)-based pressure sensors have hit a brick wall in the past due to the time-dependent phenomena experienced by piezoresisitive materials making them difficult to model and limiting the resistivity of such a pressure mapping sensor. This work investigates adding capacitive functionality to EIT-based pressure mapping by using a dielectric elastomer (DE) to aid capacitively shunt current so that not just the resistance of the device changes but also the capacitance. The hypothesis is that the non-linear time dependent effects of resistive sensing will be mitigated using this capacitive shunting.

As described in previous works \cite{Ellingham2024} there have been various attempts at created EIT-based pressure sensing devices. However, all of the state-of-the-art EIT-based pressure sensors rely on the change of resistance for pressure mapping. Zhang et. al \cite{Zhang2017} utilised capacitive shunting for EIT touch mapping on a variety of surfaces with conductive coatings. Their work used a human touch input to shunt electrical current through the person's contact point, essentially mapping localised changes in capacitance. Work from Reynolds-Smith \cite{Reynoldssmith1995,Reynoldssmith1999} showed the similarities and methods for Electric Field Tomography (EFT) as a method of localising touch and proximity of a grounded body. Another ubiquitous problem for an EIT-based pressure sensor is minaturising the EIT reconstruction processor. The smallest processors used in literature are SoCs or FPGA implementations, however these are much slower than PCs in their current state \cite{ZamoraArellano2020,Kim2017,Takhti2019,Liu2019}. With the the current developments minaturisation of parallel computing units, there are more possibilities to create an EIT-based pressure mapping sensor PCB or even ASIC that can do complete both EIT data acquisition and reconstruction tasks.

The DEA-EIT device showed potential to also act as a DEG with with the simultaneous pressure localisation function. Simulations showed a range of energy generation quantities with a range of applied loads to determine the energy generation potential of real-world loading scenarios. The next step is to modify the existing electrical hardware to run the DEG sequence and measure the losses within the system.



Dream experiments include:
- In-situ CBSR strain TEM imaging [x]
- RNNs or other non-linear modelling techniques for modelling time dependent phenomena []
- EIT with capacitive shunting AND/OR EFT-ERT MUX device [x]
- DEG experiments to validate simulation [x]
- develop an ASIC dedicated to gathering EIT data and completing inverse problems and pressure mapping co-processing?? []

% \lipsum

