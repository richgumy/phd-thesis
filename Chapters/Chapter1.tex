\chapter{\chapiname}
\label{chapter1}
The reliance on rigid robotic systems, dominated by rotary motors and conventional sensors, has long constrained the potential for innovation in robotics, limiting the adaptability and creativity needed for future technological breakthroughs. In biology, animals have many actuator units in the form of soft muscles and a multitude of various receptors within soft tissue for sensing their environment and allowing a different range of delicate control and manipulation of the natural world. The rigidity of rotational motors is stifling creativity in the creation and development of devices among other unforeseen future technology. Engineers are often constrained to solving problems and designing solutions using typical rigid sensors and actuators due to their current ubiquity and their evolved increased efficiency. With the rise of research into soft sensor and actuator devices, these soft devices will follow suit of the traditional rigid sensors and actuators and become ubiquitous, reliable, and viable options for engineering design solutions. 

This thesis has developed methods and tools for creating and characterising artificial pressure sensitive skin technology. The thesis then continues to explore the integration of this artificial skin technology into an artificial muscle technology. The work in this thesis has ultimately contributed towards filing a provisional patent for DEA-EIT actuator-sensor technology in a quest to bring this work out of the academic realms into real-world applications.



\section{Why Go Soft and Not Rigid?}
% creative motivation
The requirement for soft robotics in general has been driven by the limitations of current rigid robotic solutions to interact with natural organic material. Manipulation of natural plant and animal tissue have traditionally been handled by humans by hand due to our ability to use our dexterity and intelligent control systems to ensure minimal undesirable damage. With the advance in in technology in various soft robotic actuators\cite{Stella2023,Zhang2023,Hartmann2021,Yasa2023, Manti2016}, sensors\cite{Hegde2023}, and soft robotics control\cite{DellaSantina2023, Armanini2023}. The use of soft robotics in place of rigid alternatives, amongst other benefits, has the opportunity to be more sustainable by decreasing waste products during fabrication, using biodegradable or recyclable materials, shelf life, and use of renewable resources\cite{Hartmann2021}. The use of soft robotics brings the opportunity to create devices with a reduced bill of materials size and less moving parts for maintenance. The use of soft robotics in biomedical and aerospace applications is especially desirable due to the difficulties experienced when designing with regular motors in the outer space and near sensitive biological tissue environments such as heat dissipation, lubrication, and mass\cite{Murugesan1981,Ashuri2020,Branz2017,Bruschi2021}. 

% motivation financially
Soft sensor and actuator technology may replace many traditional rigid equivalents and be adopted by emerging technologically advanced companies requiring advanced automation and robotic handling. The most common rigid actuator market, the global electric motor market, was valued at USD 142.2 billion in 2020\cite{alliedmarketresearch}. Adjacently, the pressure mapping sensor global market value, focused mainly on the health sector, was valued at USD 480 million in 2023\cite{Bharatha2023}. These market values indicate that there could be a high value in soft robotics when a percentage of the companies in these markets were to move to using soft sensor and actuator alternatives.

% TODO: curate a list of applications that currently use rotary motors that yearn for a soft alternative!
% prosthetic limbs, space (vacuum exposed) motors, surgical implantation tools, implants, 

% why EAPs?
Soft robotic actuation can be achieved through various mechanisms including thermal, electrochemical, fluidic, magnetic, and electrostatic. Similarly soft stress-strain sensing can be achieved through various physical principles such as resistive, capacitive, magnetic, and optical sensing methods. Often the function of soft actuators can be inverted such that the deformation of the actuator can produce a signal used for self sensing, in electroactive polymer (EAP) technologies such as dielectric elastomer actuators (DEAs) \cite{Gisby2013, Rosset2013, Liu2016, Huang2023} and ionic polymer-metal composites (IPMCs) \cite{MohdIsa2019}. EAPs have the benefit of electronic control over other soft actuator and sensor technologies controlled by fluids, heat, or light which often contain the complexity and bulk of an additional energy transduction process.

% research motivation
Proprioception in artificial muscle technology has been made a reality. This is seen in the self-sensing of one dimensional strain of DEAs usually through capacitive measurement between the compliant electrodes during operations to obtain the magnitude of a contraction. However, the pressure mapping done similar to the mechanosensation performed by cutaneous mechanoreceptors on an artificial muscle device, in particular a DEA, has not been explored as of writing this thesis. 

% Reinforce novelty of the thesis
This thesis has converged on the use of conductive particle based elastomer composites and their use in sensors and actuators, in particular an electrical impedance tomography (EIT) based artificial skin and it's integration into the artificial muscle technology, dielectric elastomer actuators. The composite type used throughout the thesis is simple to fabricate but not well understood in terms of its electromechanical transient and dynamic characteristics. The modelling of such conductive particle composites would elucidate the feasibility of inverting the model to create a responsive strain sensor. This composite has been characterised in one-dimension several times in literature already however, since a two dimensional sensing application of this composite is desired the characterisation of the sensor in two dimensions was completed. A method to do such 2D sensing is using EIT. EIT has been used in the past for a huge range of applications, with few exploring the use of EIT as a pressure mapping sensor.


\section{Research Objectives}
The research objectives and questions for this thesis are given below:
\begin{enumerate}
	\item Characterise static, dynamic, and transient phenomena seen in conductive particle elastomer composites.
	\item From the characterisation above, mitigate the effects of the transient phenomena for further use in electro-active sensor and actuator technology.
	\item Develop an electrical impedance tomography based pressure sensor and validate its performance
	\item Design the hardware for sensing and characterisation of the above pressure sensor.
	\item Investigate the effects of integrating electrical impedance tomography based pressure mapping onto a dielectric elastomer actuator.
	\item Investigate energy generation impacts in the above electrical impedance tomography dielectric elastomer actuator device.
\end{enumerate}



\section{Chapter Contributions}
Chapters \ref{chapter3} - \ref{chapter7} contain the core novel research contributions. Chapters \ref{chapter2} and \ref{chapter8} provide essential background knowledge and future research directions for the thesis respectively.

\hyperref[chapter2]{\textbf{Chapter 2 - \chapiiname}}: explores the nature of biological skin and muscle from an engineering perspective, quantifying necessary functions and properties desired to replicate or supersede for their artificial equivalents. The thesis then describes state-of-the-art soft sensors and actuators and their function.

\hyperref[chapter3]{\textbf{Chapter 3 - \chapiiiname}}: uncovers the dynamic and time-dependent electromechanical tensile and compressive properties of carbon black silicone composites, in order to understand the material before it's use in sensors and actuators.

\hyperref[chapter4]{\textbf{Chapter 4 - \chapivname}}: discusses the use of electrical impedance tomography to create a pressure mapping sensor and provides tools for analysing the suitability to various applications and choosing a suitable sensing domain.

\hyperref[chapter5]{\textbf{Chapter 5 - \chapvname}}: gives the small form factor, low-cost hardware design for a EIT-based pressure mapping device.

\hyperref[chapter6]{\textbf{Chapter 6 - \chapviname}}: describes the integration of the EIT-based pressure mapping technology discussed in the previous chapters, how it can be integrated into dielectric elastomer actuators, and the trade-offs.

\hyperref[chapter7]{\textbf{Chapter 7 - \chapviiname}}: simulates and analyses the unintended power generation of the simultaneous sensor actuator device discussed in the previous chapter.

\hyperref[chapter8]{\textbf{Chapter 8 - \chapviiiname}}: summarises the key research findings of the thesis and discusses the future direction of the technology discussed in the thesis and acknowledges the future of soft robotics.

\section{Publications}
Publications towards this thesis include three conference papers \cite{Ellingham2021,Ellingham2022,Ellingham2024a}, one journal paper \cite{Ellingham2024}, one journal paper under review, and one provisional patent filed. 

%\afterpage{\blankpage}
\cleardoublepage