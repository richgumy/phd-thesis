\chapter{\chapiname}
\label{chapter1}
Rigid robotic systems often have multiple rotary motors and various sensors integrated together for precise control of the robot, this is mirrored in biology with the animals having many actuator units in the form of muscles and a multitude of various receptors for sensing their environment. The rigidity of rotational motors is stifling creativity in the creation and development of devices amongst many other unforeseen future technology. Engineers are often constrained to solving problems and designing solutions using typical rigid sensors and actuators due to their current ubiquity and their evolved increased efficiency. With the rise of research into soft sensor and actuator devices, these such device need to follow suit of the traditional rigid sensors and actuators and become ubiquitous and viable option for general and specialised engineering design solutions. 

This thesis has developed methods and tools for creating and characterising artificial pressure sensitive skin technology. The thesis then continues to explore the integration of this artificial skin technology into an artificial muscle technology.



\section{Why Go Soft and Not Rigid?}
% creative motivation
The requirement for soft robotics in general has been driven by the limitations of current rigid robotic solutions to interact with natural organic material. Manipulation of natural organic objects such as animals, plants, fruit, vegetables, and meat have traditionally been handled by humans by hand due to our ability to use our dexterity and intelligent control systems to ensure minimal undesirable damage. With the advance in in technology in various soft robotic actuators\cite{Stella2023,Zhang2023,Hartmann2021,Yasa2023, Manti2016}, sensors\cite{Hegde2023}, and soft robotics control\cite{DellaSantina2023, Armanini2023}. The use of soft robotics in place of rigid alternatives, amongst other benefits, has the opportunity to be more sustainable by decreasing waste products during fabrication, using biodegrabale or recyclable materials, shelf life, and use of renewable resources\cite{Hartmann2021}. The use of soft robotics brings opportunity of creating devices with a reduced bill of materials size and less moving parts for maintenance. The use of soft robotics in biomedical and aerospace applications is especially desirable due to the difficulties experienced when designing with regular motors in the outer space and near sensitive biological tissue environments such as heat dissipation, lubrication, and mass\cite{Murugesan1981,Ashuri2020,Branz2017,Bruschi2021}. 

% motivation financially
The most common rigid actuator is the rotary electric motor and the global market was valued at USD 142.2 billion in 2020, with a predicted growth rate of 9.5\% until 2032\cite{alliedmarketresearch}. Although this market is dominated by automobiles which currently require the traditional form of rotary electric motors, growing sectors of this market such as medical, factory automation, and aerospace have potential interest in adopting soft actuator alternatives for the reasons given above. In parallel, rigid strain sensors of types metallic foil and semiconductor, was given a global market value of USD 190.66 million in 2022 with a compound annual growth rate of 3.9\% until 2029\cite{maximizemarketresearch}. Adjacently the pressure mapping global market value, focused mainly on the health sector, was valued at USD 480 million in 2023 with an expected growth rate of 5.1\%\cite{usdanalytics,visualizeresearch}. Many soft actuator technologies could be used in these growing medical, aerospace, factory automation, and agricultural sectors.

% TODO: curate a list of applications that currently use rotary motors that yearn for a soft alternative!
% prosthetic limbs, space (vacuum exposed) motors, surgical implantation tools, implants, 

% why EAPs?
Soft robotic actuation can be achieved through various mechanisms including thermal, electrochemical, fluidic, magnetic, and electrostatic. Similarly soft stress-strain sensing can be achieved through various physical principles such as resistive, capacitive, magnetic, and optical sensing methods. Often the function of soft actuators can be inverted such that the deformation of the actuator can produce a signal used for self sensing, in electroactive polymer (EAP) technologies such as dielectric elastomer actuators (DEAs)\cite{Gisby2013, Rosset2013, Liu2016, Huang2023} and ionic polymer-metal composites (IPMCs)\cite{MohdIsa2019}. EAPs have the benefit of electronic control over other soft actuator and sensor technologies controlled by fluids, heat, or light which contain the complexity of another energy transfer process.

% research motivation
Proprioception in artificial muscle technology has been made a reality. This is seen in the self-sensing of one dimensional strain of DEAs usually through capacitive measurement between the compliant electrodes during operations to obtain the magnitude of a contraction. However, the pressure mapping done similar to the mechanosensation performed by cutaneous mechanoreceptors on an artificial muscle device has not been explored as of writing this thesis. 

Publications towards this thesis include three conference papers, one journal paper, and one provisional patent filed. This thesis has converged on the use of conductive particle based elastomer composites and their use in sensors and actuators, in particular an electrical impedance tomography (EIT) based artificial skin and it's integration into the artificial muscle technology, dielectric elastomer actuators. The composite type used throughout the thesis is simple to fabricate but not well understood in terms of its electromechanical transient and dynamic characteristics. The modelling of such conductive particle composites would elucidate the feasibility of inverting the model to create a responsive strain sensor. This composite has been characterised in one-dimension several times in literature already however, if a two dimensional sensing application of this composite is desired the characterisation of the sensor in two dimensions must be completed. A method to do such 2D sensing is using EIT. EIT has been used in the past for a huge range of applications, with few exploring the use of EIT as a pressure mapping sensor. Although EIT-based pressure mapping was first discovered 30 years ago, the technology is still in its infancy with several problems needing to be resolved before the technology can be used reliably in real-world applications.


\section{Research Objectives}
The research objectives and questions for this thesis are given below:
\begin{enumerate}
	\item Quantify and analyse static, dynamic, and transient phenomena seen in conductive particle composites.
	\item Use additive manufacturing methods and design a mixer for FDM printing of conductive particle composites for soft sensors and actuators.
	\item From the characterisation in objective 1 mitigate the effects of the transient phenomena.
	\item Create a set of metrics for quantifying the performance of an electrical impedance tomography based artificial skin.
	\item Simulate and integrate an electrical impedance tomography based artificial skin onto a dielectric elastomer actuator.
\end{enumerate}

\section{Chapter Contributions}
Chapters \ref{chapter3} - \ref{chapter7} contain the core novel research contributions. Chapters \ref{chapter2} and \ref{chapter8} provide essential background knowledge and future research directions for the thesis respectively.

\hyperref[chapter2]{\textbf{Chapter 2 - \chapiiname}}: This chapter explores the nature of biological skin and muscle from an engineering perspective, quantifying necessary functions and properties desired to replicate or supersede for their artificial equivalents. The thesis then describes state-of-the-art soft sensors and actuators and their function.

\hyperref[chapter3]{\textbf{Chapter 3 - \chapiiiname}}: This chapter uncovers the electromechanical tensile and compressive properties of carbon black silicone composites, in order to understand the material before it's use in sensors and actuators.

\hyperref[chapter4]{\textbf{Chapter 4 - \chapivname}}: This chapter discusses advanced manufacturing for 3D printing for soft actuator and sensor technology and a novel form of mixing for additive manufacturing.

\hyperref[chapter5]{\textbf{Chapter 5 - \chapvname}}: This chapter discusses the use of electrical impedance tomography to create a pressure mapping sensor and provides tools for analysing the suitability to various applications and choosing a suitable sensing domain.

\hyperref[chapter6]{\textbf{Chapter 6 - \chapviname}}: This chapter describes the integration of the pressure mapping technology discussed in the previous chapter, how it can be integrated into dielectric elastomer actuators, and the trade-offs.

\hyperref[chapter7]{\textbf{Chapter 7 - \chapviiname}}: This chapter discussed the unintended power generation of the simultaneous sensor actuator device discussed in the previous chapter.

\hyperref[chapter8]{\textbf{Chapter 8 - \chapviiiname}}: This chapter discusses the small form factor, low-cost hardware design for a hybrid artificial muscle - artificial skin based device.

\hyperref[chapter9]{\textbf{Chapter 9 - \chapixname}}: The is chapter models the a DE-EIT device in order to find an optimal range of parameters at which capacitive shunting can be used to improve the DE-EIT pressure mapping device responsiveness.

\hyperref[chapter10]{\textbf{Chapter 10 - \chapxname}}: This chapter discusses the future direction of the technology discussed in the thesis and acknowledges the future of the broad field of soft robotics.